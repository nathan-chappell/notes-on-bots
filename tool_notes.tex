\documentclass{article}

% new_commands.tex
%
% This file is meant to be \input into whatever.
%

\newcommand{\ra}{\rightarrow}
\newcommand{\la}{\leftarrow}
\newcommand{\bibtitle}[1]{{\it #1}}
\newcommand{\note}[1]{{\footnotesize #1}}
\newcommand{\hi}[1]{{\large {\bf #1}}}
\newcommand{\hii}[1]{{\it #1}}
\newcommand{\hiii}[1]{{\it #1}}
\newcommand{\M}{\mathcal{M}}
\newcommand{\file}[1]{{\tt #1}}
\newcommand{\code}[1]{{\tt #1}}


\begin{document}

\hi{Notes from Python Source}

\hii{\file{core.turn\_context.py}}
\begin{itemize}

\item \code{TurnContext}\\
    Attributes
    \begin{itemize}
        \code{adapter}: bot adapter for communication \\
        \code{activity}: is going on  \\
        \code{\_responded}: can't be set to false \\
        \code{\_services}: \note{``Map of services and other values cached for the lifetime of the turn.''} \\
        \code{\_on\_send\_activities}, \\
        \code{\_on\_update\_activity}, \\
        \code{\_on\_delete\_activity}, \\
    \end{itemize}

\end{itemize}

\hii{\file{core.dialogs.dialog.py}}
\begin{itemize}

\item \code{Dialog(ABC)}\\
    Dialogue in an \code{ABC} that has an \code{id} and \code{telemetry\_client} property, and some pretty intuitive methods.  \code{begin\_dialog} starts the dialog.  Once started, it may enter a new dialog, after which it may be resumed with \code{resum\_dialog}.  The typical interaction will be to \code{continu\_dialog} after a user response.  There are also the methods \code{en\_dialog} and \code{repromp\_dialog}.  All mentioned methods accept a \code{DialogContext}, \code{repromp\_dialog} and \code{en\_dialog} take a \code{TurnContext}, and \code{resum\_dialog} deals with the return and reason for resuming.

\end{itemize}


\end{document}
